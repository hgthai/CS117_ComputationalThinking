\chapter{THE PROCESS OF APPLYING COMPUTATIONAL THINKING}

The process of constructing a solution from a real-world problem is sequentially executed through steps of applying Computational Thinking \cite{wing2006computational} to transform raw data into structured instructions, specifically as follows:

\section{Step 1: Abstraction on the Real-World Problem}
\begin{itemize}
    \item \textbf{Real-World Input:} An image of a Water Hyacinth Weaving surface containing various noise details (color, lighting, background, physical curvature).
    \item \textbf{Process:} Eliminate unnecessary details (fiber color, minor material defects), focusing solely on the topological structure of the weaving knots. The problem is modeled as finding a mapping function $F$.
    \item \textbf{Outcome:} The real-world problem becomes a computational problem, the \textbf{Root Problem: Image-to-Instruction Synthesis}.
    \begin{itemize}
        \item \textit{Input:} Single RGB image $I \in \mathbb{R}^{H \times W \times 3}$.
        \item \textit{Output 1:} Weaving Structure Representation in the form of a Matrix $M$ (Weaving Matrix).
        \item \textit{Output 2:} Ordered Instruction Sequence $S$.
        \item \textit{Constraint:} $M$ must satisfy Structural Fidelity (R1), and $S$ must ensure Human Usability (R2).
    \end{itemize}
\end{itemize}

\section{Step 2: Decomposition on the Root Problem}
\begin{itemize}
    \item \textbf{Process:} Recognizing that a direct mapping from $I \to S$ (End-to-End) is overly complex and lacks interpretability. The problem is decomposed based on core functions: ``Visual Perception'' and ``Procedural Reasoning''.
    \item \textbf{Outcome:} The Root Problem is decomposed into two independent, sequential modules:
    \begin{itemize}
        \item \textbf{Module 1: Perception (X1) -} Transform the image into structured data.\\
        \textbullet\ \textit{Input:}  Image $I$.\\
        \textbullet\ \textit{Output:} Matrix $M$.
        \item \textbf{Module 2: Reasoning (X2) -} Plan from structured data.\\
        \textbullet\ \textit{Input:} Matrix $M$.\\
        \textbullet\ \textit{Output:} Instructions $S$.
    \end{itemize}
\end{itemize}

\section{Step 3: Decomposition on Module 1 (Perception)}
\begin{itemize}
    \item \textbf{Process:} To generate matrix $M$, the system needs to know the locations of intersection nodes and the over/under relationships at those points. Module 1 is further subdivided into specific computer vision tasks.
    \item \textbf{Outcome:} Module 1 is decomposed into 3 sequential sub-problems:
    \begin{itemize}
        \item \textbf{Sub-problem 1.1 (Local Feature Extraction):} Determine the coordinates of intersection points.\\
        \textbullet\ \textit{Input:} Original RGB Image $I$.\\
        \textbullet\ \textit{Output:} Set of intersection node coordinates $N = \{(x_i, y_i)\}$.
        
        \item \textbf{Sub-problem 1.2 (Local Relation Classification):} Determine the weaving state at each intersection point.\\
        \textbullet\ \textit{Input:} Node coordinates $N$ and local image patches around $N$ extracted from $I$.\\
        \textbullet\ \textit{Output:} Set of labeled nodes with relationships $N' = \{(x_i, y_i, r_i)\}$ (where $r_i$ is the over/under label).
        
        \item \textbf{Sub-problem 1.3 (Topology Encoding):} Encode local relationships into a global matrix.\\
        \textbullet\ \textit{Input:} Set of labeled nodes $N'$.\\
        \textbullet\ \textit{Output:} Weaving structure matrix $M$ (Weaving Matrix).
    \end{itemize}
\end{itemize}

\section{Step 4: Pattern Recognition - Matching with Sub-problems 1.1 \& 1.2}
\begin{itemize}
    \item \textbf{Process:}
    \begin{itemize}
        \item For Sub-problem 1.1: Recognize this as a problem of finding keypoints on an image.
        \item For Sub-problem 1.2: Recognize this as an image classification problem based on local visual cues.
    \end{itemize}
    \item \textbf{Outcome:}
    \begin{itemize}
        \item Sub-problem 1.1 matches the \textbf{Keypoint Detection / Object Detection} problem.\\
        $\rightarrow$ \textit{Solution:} Use CNN or ViT-based networks to extract features.
        \item Sub-problem 1.2 matches the \textbf{Image Classification} problem.\\
        $\rightarrow$ \textit{Solution:} Crop local patches around intersection points and pass them through a classification model.
    \end{itemize}
\end{itemize}

\section{Step 5: Decomposition on Module 2 (Reasoning)}
\begin{itemize}
    \item \textbf{Process:} From matrix $M$, instructions $S$ need to be generated. This requires finding a logical flow before converting to natural language.
    \item \textbf{Outcome:} Module 2 is decomposed into:
    \begin{itemize}
        \item \textbf{Sub-problem 2.1 (Procedure Planning):} Determine the execution order of weaving steps.\\
        \textbullet\ \textit{Input:} Weaving structure matrix $M$.\\
        \textbullet\ \textit{Output:} Execution Plan $P = \{p_1, p_2, \dots, p_n\}$.
        
        \item \textbf{Sub-problem 2.2 (Generate Instruction):} Convert plan $P$ into natural language.\\
        \textbullet\ \textit{Input:} Execution Plan $P$.\\
        \textbullet\ \textit{Output:} Ordered Textual Instructions sequence $S$.
    \end{itemize}
\end{itemize}

\section{Step 6: Pattern Recognition - Matching with Sub-problems 2.1 \& 2.2}
\begin{itemize}
    \item \textbf{Process:}
    \begin{itemize}
        \item For Sub-problem 2.1: Matrix $M$ is essentially a Graph or Grid representation. Finding the weaving order corresponds to traversing the nodes.
        \item For Sub-problem 2.2: Converting structured data to instruction text follows fixed grammatical rules.
    \end{itemize}
    \item \textbf{Outcome:}
    \begin{itemize}
        \item Sub-problem 2.1 matches the \textbf{Graph Traversal} problem.\\
        $\rightarrow$ \textit{Solution:} Use the \textbf{DFS (Depth-First Search)} algorithm combined with Cycle Detection to determine valid paths.
        \item Sub-problem 2.2 matches the \textbf{Rule-based Text Generation} problem.\\
        $\rightarrow$ \textit{Solution:} Use the \textbf{Template-based} method (filling in sentence templates) to ensure accuracy and comprehensibility.
    \end{itemize}
\end{itemize}

\section{Breakdown Tree}
\begin{figure}[H]
    \centering
    \includegraphics[width=0.8\textwidth]{image/tree.png}
    \caption{Breakdown Tree of the Project}
    \label{fig:tree}
\end{figure}