\chapter{Solution}

\begin{figure}[H]
    \centering
    \includegraphics[width=0.8\textwidth]{image/pipeline.png}
    \caption{Pipeline of the proposed solution}
    \label{fig:pipeline}
\end{figure}

\section{Module 1: Perception (X1) - Structure-Aware Weaving Graph Generation}

\textbf{Input:} A single RGB image $I$ depicting a water hyacinth woven surface.

\textbf{Output:} Weaving Structure Representation (Weaving Matrix $M$).

The perception module decomposes into three interconnected sub-processes, each addressing a specific aspect of structural understanding.

\subsection{Sub-problem 1.1: Local Feature Extraction}

The initial stage employs sophisticated computer vision techniques to identify and localize intersection points within the woven surface. Given the repetitive, grid-like nature of water hyacinth weaving patterns, we leverage:

\begin{itemize}
    \item \textbf{Convolutional Neural Networks (CNNs)} \cite{he2016deep, lecun2015deep} for robust feature detection capable of handling variations in lighting, texture, and perspective distortion.
    \item \textbf{Vision Transformer (ViT)-based keypoint detection} \cite{dosovitskiy2021image, vaswani2017attention} to capture long-range spatial dependencies across the woven surface.
    \item \textbf{Patch-based perception algorithms} that analyze local neighborhoods around candidate intersection points, enabling fine-grained discrimination between true structural intersections and visual artifacts.
\end{itemize}

This sub-process generates a set of intersection node coordinates:
\begin{equation}
    N = \{(x_1, y_1), (x_2, y_2), \ldots, (x_n, y_n)\}
\end{equation}
representing the foundational structural elements of the weaving pattern.

\subsection{Sub-problem 1.2: Local Relation Classification}

Once intersection points are identified, the system must determine the spatial relationships between adjacent nodes—specifically, which strands connect to form the continuous weave structure. This classification problem is addressed through:

\begin{itemize}
    \item \textbf{Graph-based reasoning algorithms} \cite{kipf2017semi} that evaluate the geometric consistency of potential connections.
    \item \textbf{Local visual cue analysis} \cite{cao2017realtime} examining strand orientation, color continuity, and texture patterns to infer connectivity.
    \item \textbf{Classification networks} \cite{redmon2016you} trained on annotated weaving samples to distinguish valid structural relationships from spurious visual alignments.
\end{itemize}

The output is a set of labeled edges:
\begin{equation}
    E = \{(n_i, n_j, r_{ij})\}
\end{equation}
where $r_{ij}$ encodes the relationship type (horizontal, vertical, or diagonal connection) between nodes $n_i$ and $n_j$.

\subsection{Sub-problem 1.3: Topology Encoding}

The final perception stage synthesizes node coordinates and relational information into a unified \textbf{Weaving Matrix} ($M$)—a structured representation that encodes the complete topological configuration of the woven surface. The encoding process involves:

\begin{itemize}
    \item \textbf{Deterministic encoding} of local relations into a normalized matrix format where each cell represents an intersection point and its associated over-under configuration.
    \item \textbf{Spatial regularization} to ensure geometric consistency across the entire weaving surface.
    \item \textbf{Topological validation} to verify that the encoded structure corresponds to a physically realizable weaving pattern.
\end{itemize}

The Weaving Matrix $M$ serves as a formal bridge between perceptual analysis and procedural reasoning, abstracting away visual complexity while preserving all structural information necessary for instruction generation.

\section{Module 2: Reasoning (X2) - Instruction Synthesis}

\textbf{Input:} Weaving Matrix ($M$).

\textbf{Output:} Ordered Weaving Instruction Sequence ($S$).

The reasoning module transforms the structural representation into executable instructions through a two-stage process that mirrors the procedural logic employed by traditional Vietnamese artisans.

\subsection{Sub-problem 2.1: Procedure Planning}

This sub-process analyzes the Weaving Matrix to determine an optimal execution strategy, addressing:

\begin{itemize}
    \item \textbf{Traversal path selection:} Determining the order in which individual cells should be addressed during instruction generation.
    \item \textbf{Cycle detection:} Identifying repetitive structural motifs that can be described efficiently through iterative instructions rather than exhaustive enumeration.
    \item \textbf{Base strand selection:} Identifying the foundational strands that must be established before subsequent weaving operations can proceed—reflecting the traditional Vietnamese practice of establishing a stable base structure (\textit{móng} or foundation) before building upward.
\end{itemize}

The algorithm employs techniques inspired by Vietnamese weaving pedagogy, including:

\begin{itemize}
    \item \textbf{Matrix-encoded topology analysis} using Depth-First Search (DFS) to establish procedural dependencies.
    \item \textbf{Pattern recognition algorithms} to identify repeating cycles (e.g., ``over-two-under-two'' patterns characteristic of many Vietnamese basket weaves).
    \item \textbf{Constraint satisfaction} to ensure that the proposed execution sequence respects physical realizability constraints (e.g., a strand cannot be woven through a position until supporting strands are in place).
\end{itemize}

The output is an ordered execution plan:
\begin{equation}
    P = \{p_1, p_2, \ldots, p_m\}
\end{equation}
representing the sequence of high-level weaving operations.

\subsection{Sub-problem 2.2: Generate Instruction}

The final stage translates the execution plan into natural language instructions comprehensible to human weavers. This involves:

\begin{itemize}
    \item \textbf{Template-based text generation} for standard weaving operations, utilizing domain-specific vocabulary consistent with Vietnamese craft terminology.
    \item \textbf{Rule-based synthesis} for complex or ambiguous configurations where template matching is insufficient.
    \item \textbf{Contextual instruction elaboration} that provides additional guidance for steps that might be challenging for novice practitioners.
\end{itemize}

The system generates instructions in both Vietnamese and English, incorporating culturally appropriate phrasing. For example, rather than abstract geometric descriptions, instructions reference tangible craft actions:

\begin{itemize}
    \item \textit{``Luồn sợi ngang qua hai sợi dọc''} (Thread the horizontal strand over two vertical strands)
    \item \textit{``Tạo móng bằng cách đặt bốn sợi song song''} (Create the base by placing four parallel strands)
\end{itemize}

The output is an Ordered Weaving Instruction Sequence:
\begin{equation}
    S = \{s_1, s_2, \ldots, s_k\}
\end{equation}
where each instruction $s_i$ corresponds to a specific, actionable weaving operation.