\chapter{Problem Definition / Clarification}
\subsection{Problem Description}
The objective of this project is to develop an automated system capable of converting a single image of a woven product (specifically water hyacinth weaving) into detailed, step-by-step weaving instructions.

\begin{itemize}
    \item \textbf{Input:} A single RGB image (JPG/PNG) of a water hyacinth weaving pattern.
    \item \textbf{Output:}
    \begin{enumerate}
        \item \textit{Weaving Structure Representation (Weaving Matrix $M$):} A matrix constructed based on cells with predefined rules.
        \item \textit{Ordered Weaving Instruction Sequence ($S$):} A textual sequence of instructions guiding the user through the weaving process.
    \end{enumerate}
\end{itemize}

\subsection{Scope}
The project focuses on close-up images of complete woven surfaces featuring regular, repetitive patterns. The aim is to generate a valid and internally consistent weaving procedure that reproduces the observed over/under topology. Recovering the original or unique artisan procedure used to create the specific sample is out of scope.

\subsection{Assumptions}
\begin{itemize}
    \item Strands are visually separable with sufficient contrast for reliable tracing.
    \item Over/under relationships at intersections are locally observable and unambiguous.
    \item Images are captured from an approximately top-down viewpoint with limited perspective distortion.
    \item All required weaving strands are available in advance, with quantities determined by the weaving matrix dimensions and sufficient lengths to execute the generated procedure.
\end{itemize}

\subsection{Constraints}
\begin{itemize}
    \item The system processes a single RGB image per instance; video or multi-view inputs are not supported.
    \item Each image contains a single homogeneous weaving pattern.
    \item The pipeline is modular and interpretable, without end-to-end image-to-instruction learning (avoiding "black box" behavior).
    \item Image resolution must ensure at least $32 \times 32$ pixels per strand intersection.
    \item The weaving surface is approximately planar, with limited perspective distortion and no severe occlusion.
    \item The image is analyzed using local patches centered at candidate intersections, with moderate intersection density to allow unambiguous local processing.
\end{itemize}

\subsection{Requirements}
\begin{itemize}
    \item \textbf{R1 (Structural Fidelity):} The generated Weaving Matrix ($M$) must seamlessly map to the physical topology of the input image.
    \item \textbf{R2 (Usability):} The system must generate valid, human-readable weaving instructions.
\end{itemize}

\newpage