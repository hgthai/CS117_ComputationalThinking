\chapter{PROBLEM IDENTIFICATION}

\section{Problem Introduction}
Water hyacinth (Eichhornia crassipes) weaving represents a quintessential Vietnamese traditional craft that embodies centuries of ecological wisdom and cultural ingenuity. Originating primarily in the Mekong Delta region and various Vietnamese rural communities, this artisanal practice transforms an abundant aquatic plant into intricately woven handicrafts, ranging from household items to decorative artifacts. The weaving technique itself reflects the Vietnamese philosophy of "đảo ngược tai họa thành phước lành" (transforming misfortune into blessing), as artisans convert an invasive aquatic species into economically valuable products while simultaneously addressing environmental concerns.
\begin{figure}[H]
    \centering
    \includegraphics[width=0.8\textwidth]{image/intro.png}
\end{figure}

The craft of water hyacinth weaving is deeply interwoven with Vietnamese cultural identity and rural livelihoods \cite{kurin2021my}. Traditional artisans, predominantly women in villages such as Tân Phú Đông (Tiền Giang Province) and various communes in An Giang and Đồng Tháp provinces, have passed down sophisticated weaving patterns through generations via oral transmission and apprenticeship-based learning. These patterns often incorporate symbolic motifs drawn from Vietnamese agrarian life, natural landscapes, and spiritual beliefs, creating products that serve not merely utilitarian purposes but also function as cultural artifacts carrying intangible heritage values \cite{terras2016enabling}.

However, this invaluable traditional knowledge faces unprecedented threats in the contemporary era. The transmission mechanism of weaving expertise remains largely dependent on direct master-apprentice relationships and familial inheritance, making it vulnerable to disruption from socioeconomic transformations. As younger generations increasingly migrate to urban centers pursuing alternative livelihoods, the continuity of this craft tradition becomes precarious. Furthermore, the tacit nature of weaving knowledge—embedded in muscle memory, spatial reasoning, and experiential understanding—poses significant challenges for documentation and preservation through conventional means.


\section{Problem Description}
The objective of this project is to develop an automated system capable of converting a single image of a woven product (specifically water hyacinth weaving) into detailed, step-by-step weaving instructions. This approach is inspired by recent advances in neural inverse manufacturing, particularly the work on converting knitting images to machine instructions \cite{kaspar2019neural}.

\begin{itemize}
    \item \textbf{Input:} A single RGB image (JPG/PNG) of a water hyacinth weaving pattern.
    \begin{figure}[H]
    \centering
    \includegraphics[width=0.4\textwidth]{image/input.png}
    \end{figure}
    \item \textbf{Output:}
    \begin{enumerate}
        \item \textit{Weaving Structure Representation (Weaving Matrix $M$):} A matrix constructed based on cells with predefined rules.
        \item \textit{Ordered Weaving Instruction Sequence ($S$):} A textual sequence of instructions guiding the user through the weaving process.
        \begin{figure}[H]
        \centering
        \includegraphics[width=0.8\textwidth]{image/output.png}
        \end{figure}
        \begin{figure}[H]
        \centering
        \includegraphics[width=0.8\textwidth]{image/2.png}
        \end{figure}
        \begin{figure}[H]
        \centering
        \includegraphics[width=0.5\textwidth]{image/1.png}
        \end{figure}
    \end{enumerate}
\end{itemize}

\section{Scope}
The project focuses on close-up images of complete woven surfaces featuring regular, repetitive patterns. The aim is to generate a valid and internally consistent weaving procedure that reproduces the observed over/under topology. Recovering the original or unique artisan procedure used to create the specific sample is out of scope.
\section{Assumptions}
\begin{itemize}
    \item Strands are visually separable with sufficient contrast for reliable tracing.
    \item Over/under relationships at intersections are locally observable and unambiguous.
    \item Images are captured from an approximately top-down viewpoint with limited perspective distortion.
    \item All required weaving strands are available in advance, with quantities determined by the weaving matrix dimensions and sufficient lengths to execute the generated procedure.
\end{itemize}

\section{Constraints}
\begin{itemize}
    \item The system processes a single RGB image per instance; video or multi-view inputs are not supported.
    \item Each image contains a single homogeneous weaving pattern.
    \item The pipeline is modular and interpretable, without end-to-end image-to-instruction learning (avoiding "black box" behavior).
    \item Image resolution must ensure at least $32 \times 32$ pixels per strand intersection.
    \item The weaving surface is approximately planar, with limited perspective distortion and no severe occlusion.
    \item The image is analyzed using local patches centered at candidate intersections, with moderate intersection density to allow unambiguous local processing.
\end{itemize}

\section{Requirements}
\begin{itemize}
    \item \textbf{R1 (Structural Fidelity):} The generated Weaving Matrix ($M$) must seamlessly map to the physical topology of the input image.
    \item \textbf{R2 (Usability):} The system must generate valid, human-readable weaving instructions.
\end{itemize}

\newpage